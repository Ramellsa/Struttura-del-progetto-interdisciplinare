\section{Le rôle des multinationales chinoises sur le territoire national}
En ce chapitre nous nous concentrerons sur l’analyse de l’évolution et le rôle des entreprises chinoises dans la période de la globalisation. Les multinationales chinoises ont un poids très révélant dans le monde et dans la période de la IVème révolution industrielle ou les entreprises sont les principales actrices. Les objectifs sont d’analyser les données économiques et politiques chinoises qui permettent de contextualiser la situation chinoise et comprendre les actions faites par les plus grandes entreprises pour devenir multinationales. 
\subsection{Chine, l'Eldorado des multinationales étrangères}
Pendant les dernières 30 années, le PIB de la Chine a augmenté de 3\% à 18\% par rapport à l’Union Européenne et aux Etats-Unis qu’a diminués le PIB. Dans le passé UE et USA représentaient la moitié de la production mondiale mais aujourd’hui la Chine en dépassant tous les deux. 
Dans les régions rurales de la Chine le PIB agrégé a augmenté de 25\% et dans les plus grandes villes chinoises de 50\% dans les années au tour de 1980.  Il y avait quand même un taux de pauvreté très élevé avec des personnes qui vivaient avec 1,9\$ par jour, cela représentait 80\% de la population, mais en 2018 c’est plus au moins 10\%. La Chine a évolué beaucoup du Maoisme à aujourd’hui. Cependant, la croissance économique a ralenti en 2010 ; en effets, on a eu un ralentissement de productivité, d’investissements et du marché actionnaire avec un «boom» de la dette publique et privée (Shadow banking). 
De 1978 à 1993 la productivité du travail s’est accrue de 7\% par an dans les entreprises chinoises et 50\% de cette croissance étaient liés à la hausse de la productivité globale.

A la figure (no) est déportée l’intensité capitalistique et la productivité du travail en Chine entre 1980 et 2004. On peut observer que l’intensité capitalistique pendant les années a eu une croissance exponentielle surtout après les années 2000 où selon Justin Lin, 1 milliard de yuans investis dans les entreprises d’État créent 100 000 opportunités d’emplois, alors que cet investissement engendre 5 fois plus d’emplois dans les entreprises non étatiques.  La multiplication des investissements était la majeure responsable de surproduction en Chine.  Cette nation voulait avoir une croissance très grande et rapide et pour ça avec trop des investissements et de production elle a eu aussi beaucoup des dettes.
Les investissements physiques, donc des immobiles et des machines de production, avec la réallocation sont loin de la frontière, où le gouvernement utilise les technologies déjà en utilisation par rapport au passé où il exploitait des capitaux de provenance étrangère. Maintenant pour les régions situées près de la frontière l’Etat vise à des politiques d’éducation, aux capitaux humains et à l’innovation. Elles sont les régions des ZES ou il y a la liberté du commerce.
La Chine depuis les années ’90 a eu une augmentation des investissements sur le territoire très important. Les investissements publics en 1997 représentaient 40\% de l’investissement total de la Chine et sur le territoire il y avait 17\% des entreprises publiques. Le pays avait une politique spécifique de croissance, d’abord avec des politiques qui ont aidé les entreprises «insiders», après avec des politiques qui ont aidé les start-up, en utilisant le modèle de Silicon Valley. Pour cette raison-là il y a eu un développement financier.
Utiliser le modèle de Silicon Valley ça veut dire qu’avec les nouveautés technologiques chinois a Shenzen, avec des inventions et fabrication les exportations sont le 90\%. La Chine a créé un nouveau modèle sur le modèle de Silicon Valley, elle a utilisé internet et la vente online comme une moyen de communication et publicitaire extraordinaire, et du fait, ces produits sont arrivés à tout le monde. 
Mais il faut expliquer toute la situation à l’avance.
Les réformes économiques de 1979 et de1997 sont liées à la décollectivisassions de l’agriculture et de la politique industrielle. La privatisation ayant lieu dans les années ’90 a contribué à une croissance salariale limitée par rapport à la productivité . 
En 1780 avec les réformes agricoles la Chine, avec la production du sucre, produisait 2,27 millions de tonnes du sucre. Il y a eu une augmentation importante à moyen terme.

Avec le marché du sucre le pays a eu des entrées de capitaux significatives, grâce aussi aux accords avec l’UE et l’OMC qui ont favorisées l’exportation du sucre.
Les années 1980 sont les années de réformes chinoises. Il y a eu une réorganisation administrative qui a rétablie l’ordre et l’autorité dans un contexte chaotique. La centralisation du pouvoir, effectuée à l’aide d’une restauration économique rapide, a permis à la Chine de viser le marché international. De cette façon aussi le gouvernement a pu être plus efficace et organisé.  
Avant les fonctions du gouvernement et des entreprises était mêlées mais avec ce nouvelle approche on voulait séparer les deux sujets. 
En 1997 les représentants du Parti communiste se réunissent aux15ème Congrès à Pékin où les délégués ont affirmé le principe de l’ouverture de marché aux leur capital. Ce principe a permis aux entreprises d’Etat de vendre tout ou une partie de leur capital en permettant de financier la modernisation technologique. A partir de ce moment-là en Chine favorise la privatisation des entreprises. Après on a commencé une transformation des entreprises d’Etat en société par actions. Comme ça il y eu une accélération des réformes grâce à cette politique .
Parallèlement avec l’admission à l’OMC la Chine a obtenu une libération de son industrie sucrière qui a favori l’industrie locale.
L’approche de l’Etat se basait sur la tolérance ; en effet Chine était un pays tolérant pour   les entreprises privées, mais avec des moyens semi légaux. Pour la création d’une entreprise sur le territoire il fallait   au moins 270 jours mais avec les contacts corrects on avait le support de l’Etat. La corruption joue un rôle important dans cette histoire.
Cette politique a ouvert des grandes perspectives pour les investisseurs étrangers. Le gouvernement chinois a affirmé que la vente d’une partie des actifs n’a pas empêché de maintenir le contrôle sur ses entreprises et donc il y a eu une privatisation, mais l’Etat voulait quand même maintenir une partie de contrôle.
La Chine a dû nécessairement affronter la thématique de la modernisation économique. Comme on lit dans Les Etudes du CERI - n° 37 - janvier 1998 « D'un point de vue formel, force est d'admettre que la transformation d'entreprises en sociétés par actions n'induit pas (ou pas forcément) une privatisation »  Ça signifie que les règlements ont en realité seul limité l’ouverture aux capitaux privés. Les petites entreprises ont été partiellement privatisées. D’un côté les entreprises côtés en bourse ont été en part privatisées et de l’autre l’Etat a maintenu une partie. 
En Chine il y avait donc le phénomène des surinvestissements et pour le résoudre on a décidé d’agir sur le taux d’intérêt. Comme proposait Nicholas Lardy en 2006 il fallait relever la consommation nationale en réduisant le taux d’épargne. Cela a causé l’utilisation d’une politique visant    à modifier la demande nationale.
Pour ce faire, il s’agirait de baisser le taux d’épargne, en diminuant les impôts, et augmenter les salaires qui de conséquence accroitraient aussi la dépense publique. 

La solution d’agir sur le taux d’intérêt était celle de relever le taux et donc rendre moins attractif les investissements. 
On peut dire que les investissements chinois ont été excessifs, beaucoup d’entreprises publiques ont reçues des prêts qui ne sont pas capables de rembourser. La Chine après les réformes a investi dans la production et aussi dans la construction.  On peut voir comment les infrastructures chinoises sont accrues en manière exponentielle dans les années avec des bâtiments géants par rapport aux autres continents et il faut dire que la majorité des investissements est public.

Dans les années 2000 tout en Chine a subi une explosion radicale. Le graphique montre comment le rendement des actifs, avec les investissements dans le stock et aussi avec l’augmentation des amortissements, et des actions ont augmentées.
Les investissements de l’Etat et des entreprises d’Etat ne sont pas étés si rentables par rapport aux investissements des entreprises privatisées. Les banques publiques ne prétendaient pas des taux d’intérêt excessifs des entreprises privées. La restitution par les entreprises publiques était plus difficile. Ça clairement a augmenté la dette publique. 
Après le 2005 avec la participation des banques étrangères le surinvestissement a subi une limitation qui a permis d’assainir la situation financière.
Cependant la Chine grâce aux changements politiques, sociaux et économiques est devenue une nation très attractive pour les pays étrangers qui ont décidé de délocaliser leurs filiales dans le pays asiatique. Les grades firmes transnationales ont investi en Chine et elle a eu une entrée des capitaux importante. Il y a eu un cycle d’argent qui a permis d’avoir un mécanisme d’ajustement pour l’économie chinoise. Grâce aux mutations technologiques, les NTIC, on a vécu une croissance globale au niveau pas seulement chinois mais aussi mondial.
Par exemple des entreprises comme Carrefour ou l’Oréal ont délocalisées en Chine pour la situation et le contexte chinois. Surtout Carrefour a créés des écoles de gestion à Shanghai et avec cette stratégie aussi la Chine a pu gagner de cette situation en appendissent des techniques occidentales et devenir avec un mélange des techniques en différents domaines une puissance mondial. 


\subsection{Les entreprises locales qui investissent}

Après les investissements locaux en Chine avec la délocalisation des entreprises dans le ZED comment on a déjà expliqué avant, le pays asiatique a investi mondialement.
La Chine, comme par exemple l’Inde, aujourd’hui est vue comme un pays d’une économie émergente. Les entreprises ont concentré leurs investissements aussi à l’étranger, en Europe, comme dans le reste du monde. La présence économique chinoise en Europe inclut aussi les placements des fonds souverains de l’Etat ainsi que l’ouverture d’infrastructures par les grandes banques publiques chinoises.  
L’Europe est la première zone d’accueil mondiale pour les investissements étrangers, et la zone de premier plan comme partenaire commercial de la Chine où les flux de commerce international et d’investissements sont directs.
Le profil chinois est fortement polarisé, L’Asie avec Hongkong comme destination privilégiée des IDE de l’Europe. Les premières entreprises chinoises en Europe provenaient d’Hongkong, en 1980. Une époque où les ambitions internationales se limitaient aux entreprises publiques. Derrière ces investissements d’acquisition des actions plus que d’acquisition des entreprises, il y avait des compagnies financières et des holdings, compagnies de transport, de télécommunication et entreprises de textile qui ont eu des revenues très importantes pendant les années 2000 et qui veulent investir aussi au niveau international. Principalement les investissements étaient localisés en Royaume-Uni avant 1997. 
Les investissements réels commencent après 1990 de la part des secteurs bancaires, services et ressources naturels. L’exemple de la Chine Bank à Paris en 1986 c’est la preuve que quelque chose est en train de changer. 
Après 2000 les investissements chinois en Europe changent de dimension, stimulés à la fois par l’ouverture de la Chine et l’adhésion à l’OMC. Mais en Europe il y a eu les effets de la crise de 2008 et un contexte économique incertain né en USA. Comme ça les investissements virent sur des zones géographiques riches des matières premières comme l’Afrique, l’Amérique latine et l’Australie. 
Les motivations de l’intérêt au dehors de la Chine des multinationales et des entreprises chinoises sont surtout le retard technologique et l’expérience qui manque dans la coordination et le contrôle des filiales étrangères. 
Les motifs des investissements consistants au dehors de la Chine sont l’accès aux matières premières, accès au marché des exportations, la création d’actifs et de compétence et la recherche d’efficience dans la productivité. 
Par exemple pour l’accès à des ressources naturelles, Sinopec, une entreprise de qui vente pétrole, a fait un rachat de 47\% du groupe suisse Addax, entreprise du même secteur, pour 5 milliards de dollars. Ce rachat est significatif de la volonté chinoise d’être capable de viser au marché global. Les zones d’activité étrangère de la Chine sont donc très vastes, de l’Afrique à l’Amérique du Sud.  
Il faut dire que la pénétration des entreprises chinoises en Europe n’est pas si facile, beaucoup de marques même connues en Chine n’ont pas de grande connaissance en Europe, donc il y a une limitation des firmes qui peuvent se faire connaitre dans le monde.
Les modalités d’entrée sont différentes, il y a les entreprises qui ont été achetées par des investisseurs chinoises, ou autres qui ont été créés en Europe avec des fonds chinoises. La deuxième est parfois le meilleur parce qu’elle permet aux entreprises de faire une période d’apprentissage avec une connaissance graduelle de la méthode occidentale.
Depuis l’accord avec l’OMC les IDE sont multipliés et de conséquence aussi les achats des entreprises européennes. Ces acquisitions sont dans la majorité des cas de provenances privées et pas publiques. Cela permet de certifier la tolérance de la Chine vers la privatisation des entreprises et la volonté de croissance. 
Les domaines d’activités principales des chinois en Europe sont le secteur technologique et électrique, des automobiles et des NTIC. On peut voir en Suisse comment des entreprises de luxe et de chimie-biologie sont de propriété chinoise.
Grâce aux avantages politiques suite au changement des organes administratifs la Chine a pu devenir un pays relevant dans l’économie globale. La Chine est devenue un exportateur de produit de manufacture avec des caractéristiques pas seulement sophistiquées mais aussi de nature simple et avec un coût très bas.   
Les multinationales chinoises sont devenues importatrices des matières premières en utilisant les transports par voie maritime. Elle est devenue le premier importateur de la plupart des matières comme le coton, le caoutchouc naturel et le soja. 
Elle est aussi active dans le domaine des énergies, avec les hydrocarbures et de conséquence le pétrole.
Il faut dire que la plupart de la production chinoise est destinée à l’exportation par voie maritime . Les investissements au dehors du pays sont faits dans le secteur des industries. 
La Chine ne produit pas véritablement beaucoup de matières premières, elle import et après vende. 

\subsection{Les rapports commerciaux dans le territoire chinois}
Comment on a déjà souligné la Chine après l’adhésion à l’OMC a pu faire partie de l’économie globale et aussi de faire des échanges au niveau mondial. Ça a permis une croissance exponentielle dans le ranking des puissances mondiales jusqu’à arriver à la première place.  
La Chine avec l’OMC a fait plusieurs accords sur des divers domaines comme par exemple les accords sur les importations, sur le commerce, développement et l’évaluation en douane tous depuis le 2004, qui sont très importants pour l’économie et le commerce.
Le programme fait par la Chine sur l’accès aux marchés en franchise de droits et sans contingent pour les pays moins avancés permet d’avoir un traitement tarifaire préférentiel non réciproque. Donc ça comporte l’aide de l’OMC aux PMA. La Chine peut, depuis 2010, bénéficier des lignes tarifaires qui statistiquement correspondent au 97\% de l’ensemble des lignes tarifaires de la Chine. Donc les exportations des PMA vers la Chine en 2010 ont une valeur de 25,5 milliards de dollars. 
Une asse très importante de fluxes des capitaux en Chine c’est celui avec l’Afrique. Le site internet des rapports entre Chine et Afrique appelle le rapport comme une nouvelle rue de la Soie. Le pays asiatique investisse plus de 3 milliards de dollars par année en Afrique ma la raison principale de cette relation c’est la volonté chinoise de délocaliser la production dans le continent africain. Les rapports les plus étroits était jusqu’à 2013 avec le Sénégal mais dans les dernières années avec l’augmentation de la dette publique sénégalais les investissements ont diminué. 
Clairement avec les changements administratifs et politiques au niveau de l’Etat, aussi les entreprises sont changées. Avec le début de la libération économique de Deng Xiaoping les entreprises ont commencé à investir. L’objectif c’était d’accroitre l’influence de la Chine dans le contexte international. La politique qui a permis d’accélérer ce procès était celle du « Go global » dans les années 2000. Initialement était vise aux sociétés d’Etat mais après aussi aux entreprises privées. Grâce au soutien des gouvernements centraux, après les reformes administratives, et aussi aux activités diplomatiques intensives et enfin des prêts à long terme, la Chine a eu un accroissement des exportations et le renforcement des entreprises d’Etat chinoises. Tous les rapports commerciaux concernent aussi la recherche des ressources naturelles ainsi que l’élimination des barrières fiscales relatives aux investissements directs ont permis l’évolution chinois. 
Du fait qu’en chine la main d’ouvre coutait très peut, la productivité nationale était haut. Ça rende la chine attractive pour tous les pays étrangers, mais elle a aussi investi sur le NTIC et avec un accord très important avec ANASE. Cette relation a portée à la nation des moyens financiers importants, la croissance de la banque nationale chinois, des conséquentes investissements et en général la réalisation des projets avec le but de financier l’énergie et la télécommunication.  
Toutes ces relations commerciales ont été permises de les ZES des zones très innovatrices.
Actuellement en 2019 les entreprises chinoises doivent être soumises à des nouvelles lois internationales. Lois sur la sécurité sociale, sur le commerce électronique et la réglementation du commerce électronique transfrontalier, sur la pollution des sols, se protection de la propriété intellectuelle, sur les réductions tarifaires dans le cadre de l’ALE et enfin sur le zéro droit de douane sur les marchandises de Hong Kong. 

\subsection{D'entreprises à multinationales: un grand changement}
Les entreprises chinoises après avoir consolidé leurs brands sur le territoire national et donc avoir acquis une solidité ont décidé de se développer au niveau mondial.
 Elles ont utilisé les alliances des joint-ventures pour accéder aux technologies des multinationales occidentales. Depuis le 1996 plus de 270000 entreprises en JV sino-étrangères, que ça veut dire constitué en Chine avec capitaux étagères. Cette relation a permis de connaitre les techniques occidentales de savoir-faire industriel et leurs technologies. Donc par exemple VW qui a pu être en Chine comme JV elle pouvait détenir seulement moins de 50\% du capital social .
Ces relations ont permis d’avoir des contacts avec l’étranger et d’investir en différents pays. Huawei par exemple a créé deux laboratoires de recherche et développement en Texas avec Texas Instruments pour ce développer et créer technologies propres.
Avant avoir investi en Amérique latin et Afrique pour les matières premières ces grandes entreprises a fait une pénétration des marchés étrangers pour acquérir technologies, capacités de management, informations sur les autres marques.  
Au lieu d’investir dans la consolidation du nom d’une marque chinois dans le monde la majorité des entreprises chinoises décidées de procéder avec un processus de fusion-acquisition. Huawei en 2005 a obtenu le rachat du pôle PC de la FMN américaine IMB. Avec cette stratégie on peut avoir l’avantage de bénéficier de relations locales déjà eu. Huawei s’est implanté aussi sur le marché russe de Télécom avec Beto-Huawei et ce processus a été fait par plusieurs entreprises chinoises dans tout le monde. 
Du marché du pétrole en Angola a ce des technologies en USA et Europe la Chine a été capable d’accélérer le temps d’expansion général et se globaliser sans égales. L’argent chinois en 2019 est derrière aux majeures business mondial, aussi dans le football, voie l’acquisition du club italienne Internazionale  ou A.C. Milan . La Chine c’est un marché intéressant pour les pays étrangers et aussi l’étranger est attractif pour la Chine, donc l’expansion et la croissance chinois c’est une question de fait.

\subsection{Les partenaires commerciales des multinationales}
Les entreprises chinoises après avoir consolidé leurs brands sur le territoire national et donc avoir acquis une solidité ont décidé de se développer au niveau mondial.
 Elles ont utilisé les alliances des joint-ventures pour accéder aux technologies des multinationales occidentales. Depuis le 1996 plus de 270000 entreprises en JV sino-étrangères, que ça veut dire constitué en Chine avec capitaux étagères. Cette relation a permis de connaitre les techniques occidentales de savoir-faire industriel et leurs technologies. Donc par exemple VW qui a pu être en Chine comme JV elle pouvait détenir seulement moins de 50\% du capital social .
Ces relations ont permis d’avoir des contacts avec l’étranger et d’investir en différents pays. Huawei par exemple a créé deux laboratoires de recherche et développement en Texas avec Texas Instruments pour ce développer et créer technologies propres.
Avant avoir investi en Amérique latin et Afrique pour les matières premières ces grandes entreprises a fait une pénétration des marchés étrangers pour acquérir technologies, capacités de management, informations sur les autres marques.  
Au lieu d’investir dans la consolidation du nom d’une marque chinois dans le monde la majorité des entreprises chinoises décidées de procéder avec un processus de fusion-acquisition. Huawei en 2005 a obtenu le rachat du pôle PC de la FMN américaine IMB. Avec cette stratégie on peut avoir l’avantage de bénéficier de relations locales déjà eu. Huawei s’est implanté aussi sur le marché russe de Télécom avec Beto-Huawei et ce processus a été fait par plusieurs entreprises chinoises dans tout le monde. 
Du marché du pétrole en Angola a ce des technologies en USA et Europe la Chine a été capable d’accélérer le temps d’expansion général et se globaliser sans égales. L’argent chinois en 2019 est derrière aux majeures business mondial, aussi dans le football, voie l’acquisition du club italienne Internazionale  ou A.C. Milan . La Chine c’est un marché intéressant pour les pays étrangers et aussi l’étranger est attractif pour la Chine, donc l’expansion et la croissance chinois c’est une question de fait.

La Chine exporte en Afrique pour le 95\% biens manufacturés, il s’agit principalement de machines et d’équipements de transport. Au contraire, elle importe pétrole et minerais.
\subsection{Les pays concurrents de la Chine}
Dans un monde globalisé ou les transactions des matières premières peuvent être entre plusieurs nation même d’un côté à l’autre du monde il y a des joues d’équipe entre superpuissances et petits pays. C’est clair qu’en 2019 les deux superpuissances mondiales sont la Chine et les Etats Unis. 
Le pays qui fait une concurrence directe à la Chine est donc les USA. Le match se joue sur plusieurs aspectes. La Chine détient le record du monde en matière de dépôt de brevets avec 2,3 millions mais celles-ci qui ont été déposté a l’Organisation internationale de la propriété intellectuelle sont 21 000 en 2013, elle est très active dans la recherche des nouvelles idées et cette donnée respect l’objective de cette pays. A la deuxième place il y a tout à fait les USA avec plus ou moins 57 000 brevets en 2013 que si on regard effectivement les brevets présentés aux OMPI sont à la première place. Tous les deux pays veulent se concentrer et investir dans le développement expérimental plus que de la recherche fondamentale.  Cette a dire que par exemple en Chine le Parti communiste ne limite plus les découvertes et donc l’Etat favorise le développement des projets scientifiques avec des aides financières.
Le professeur de sciences politiques Par Minxin Pei a définit la concurrence entre les deux superpuissances une guerre froide commercial. Un facteur de cette incompatibilité c’est la fragmentation économique et géopolitique sur le secteur technologique. Il y a des normes qui donnent solidité à cette concurrence.   
Il faut dire qu’après la crise de 2008 les USA ont perdue place dans le ranking des puissances mondiales.  Maintenant Pékin est devenu roi du commerce mondial.
Un exemple relevant de la capacité de la Chine de ne pas seulement dépasser les Etats Unis avec ces inventions mais aussi de démontrer la capacité d’améliorer les découvertes américaines est le modèle du Toyotisme qui a remplacé le fordisme dans la production automobilistique. Le Toyotisme a ajouté au Fordisme la personnalisation dans la production des autos.
Pas seulement les américaines sont en concurrence avec les chinois, aussi l’Inde et la Corée se sont mise en concurrence dans le commerce qui comprend le territoire africain. 
Entre 2003 et 2012 l’Inde a dépassé la Chine en Afrique avec les investissements avec des capitaux totaux de 52 milliards des dollars. Elle a dépassé en termes d’énergies fossiles, produits chimiques et métaux.
La Corée depuis 2000 a eu une croissance, elle a établi 25 milliards des dollars en 2011 en devenant partenaire commerciale de l’Afrique principalement dans les ventes de matériel ferroviaire.
La Chine a aussi une concurrence interne dans le contexte asiatique, le Japon. Tous les deux sont des puissances économiques offrant des modèles différents et elles ont ambitions mondiales. Le Japon a un modèle d’économie libérale tant dit que la Chine d’économie dirigée. Les japonaises sont des concurrentes pour la maîtrise des matchés et cette rivalité a une connotation historique depuis le 1937-1945.
