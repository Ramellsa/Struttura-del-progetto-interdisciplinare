\section{D’une économie planifiée au socialisme de marché}
Ce premier chapitre du travail de recherche vise à expliquer pourquoi et comment a évoquée la situation économique de la Chine. Après une contextualisation de l’histoire du commerce chinois et une exposition de la situation économique du pays, sont expliquée les idées d’un politique qui veut moderniser la Chine, pays pauvre et arriéré. Sont aussi expliqués les moyens qu’il a utilisés pour rendre la Chine un pays moderne et attractif et successivement sont exposés les conséquences de ce développement économique extraordinaire. En fin, il est pris en considération le cas de Huawei, une petite entreprise qui devienne une multinationale leader du secteur des technologies de l’information et de la communication (TIC). Le but est de représenter au moyen de Huawei comment le développement technologique et économique a permis la naissance des multinationales en Chine
\subsection{1978 : l’année clé du changement en Chine L’introduction des réformes économiques en 1978}
Déjà vers 1600-1700 la Chine est un pays exportateur qui a un rôle important dans les échanges mondiaux. Elle commerce avec les pays occidentaux en exportant surtout de la soie, du thé, des porcelaines et des produits caractéristiques orientaux.  La Chine se présente comme un pays plutôt autosuffisant grâce à son agriculture et pour cela elle ne nécessite pas d’importations. Plus tard elle se spécialise dans la production des produits textiles et manufactures. Avant la première révolution industrielle l’Asie est considérée comme la première zone manufacturière du monde. En effet en 1750 la Chine et l’Inde ensemble produisent 57.3\% des manufactures au niveau mondial et l’Asie est la source de 80\% du produit national brut de la planète.  En général, les échanges commerciaux entre Orient et Occident favorisent plus l’Asie que ne nécessitait pas le produits européens par rapport à l’Europe qui se démontrait plus intéressée aux produit orientaux.  Mais la situation change et l’économie chinoise connait un ralentissement au cours des années 1800-1950 qu’est dû à tensions internes et guerres.  Un exemple est la guerre contre l’Angleterre pour interdiction de l’importation en Chine de l’Opium, une substance stupéfiante, qui détruit la Chine. La Chine doit aussi affronter des guerres contre le Japon pour la possession des certains territoires.  C’est pour cela que pendant les années 1800-1950 l’économie chinoise connait une période d’affaiblissement dans un monde où le progrès économique connait une forte accélération . En Europe la production et le développement général subissent des accélérations dues aux révolutions industrielles que se déroulent.  En effet la production mondiale se multiplie de huit et le revenu mondial par habitant par 2,6.  Le revenu mondial par habitant se multiple par 8 dans les Etats-Unis, par 4 en Europe et par 3 en Japon, mais en Chine il subit un ralentissement exactement comme le produit intérieur brut.  A côté des désordres internes du pays qui créent des tensions et arrêtent la modernisation, il y a aussi la souscription aux traités inégaux  qui ne permette pas à la Chine de bénéficier du commerce et elle rallent l’avancement de l’économie.  En plus, la Chine, qui se présente comme un pays peu avancé et faible par rapport aux pays qui ont bénéficié des innovations techniques, doit faire face à l’intrusion coloniale étrangère.  Entre 1820 et 1950, le revenu par tête en Asie, c’est-à-dire 2/3 de la population mondiale, possède in taux annuel de croissance de 0,2\% ; 4,5 fois moins que la moyenne mondiale et 6 fois moins que l'Europe occidentale. Sur l'ensemble de la période 1820-1950 la croissance des pays européens où a eu lieux la Révolution Industrielle est d'environ 400\%, celle de la Chine est seulement du 17\%.  Le pouvoir en Chine a toujours été dans les mains des différentes dynasties mais en 1911, grâce au mouvement nationaliste, s’affirme la République que met fin aux pouvoir des dynasties et au régime feudal. En Chine il y a principalement deux grouppes politiques : les communistes et les nationalistes et entre les deux il y a des disputes. Jusqu’ quand le grouppe communiste commence à gagner du pouvoir et en 1921 Mao Zedong fonde le parti communiste et ensuite en 1949 la République populaire de Chine.  Au fil du temp, en Chine se déroule une révolution communiste conduite par Mao Zedong qui propose un modèle économique socialiste  à planification centrale qui suive le modèle soviétique  et rejet le capitalisme.   Le parti communiste prend toujours plus de pouvoir et il commence une politique d’économie planifiée  qui comporte la collectivisation des terres, l’étatisation des entreprises, le rationnement de la consommation et le contrôle du taux d’intérêt.  Grâce à tout cela, il y a des améliorations dans la santé publique et l’instruction.   La tentative de Mao d’améliorer l’économie chinoise a initialement succès, c’est-à-dire que le PIB subit une triplication et le revenu par habitant augmente de l’80\%.  Mais successivement, le système ne fonctionne plus et  la Chine vit un période de stagnation de la production, une croissance de la population et la mort causé par la faim de près que 30 millions de personnes.  Tout ça risque de faire exploser une guerre civile  et la Chine reste un pays pauvre et sous-développé et sa croissance économique est pratiquement nulle en 1976  quand Mao meure. La Chine est un pays relativement isolé de l’économie mondiale, qui est en plein essor, en effet, les échanges avec l’étranger et les investissements étrangers en Chine diminuent.  
Le siècle du ‘900 et la politique de Mao sont expliqués en détails dans le chapitre 2, ici ils servent seulement en tant que contextualisation. 
\subsection{1978 : l’année clé du changement en Chine}
En ce chapitre on explique comment Deng Xiaoping réussit à rejoindre le pouvoir nécessaire pour mettre en pratique ses idées qui représentent une nouveauté dans un pays qui est emprisonné dans l’idéologie communiste très stricte. Ils sont aussi expliqués la réforme qu’il propose pour faire renaitre l’économie chinoise et les immédiates conséquences de ces changements
\subsubsection{Deng Xiaoping : « peu importe que le chat soit noir ou gris pourvu qu’il attrape les souris »}
L’écriture de ce sous-chapitre est faite en prenant le documentaire « Deng Xiaoping, l’enfance d’un chef » comme source principale. 
Deng Xiaoping participe à l’âge de 14 ans à un programme spécial de travail-étude qui prépare les jeunes chinois aux études en France, dans le but de former les futures élites du pays chinois. L’échange permette d’instruire les jeunes au capitalisme  français pour pouvoir apporter les connaissances apprises en Chine. Deng va à Paris et il commence son parcours d’étude, mais l’agence qui organise les échanges fais faillite et il ne peut pas continuer ses études. Il commence à prouver d’intéresse pour le communisme et il devienne membre des certaines associations politiques communistes. Plus tard il se dirige à Moscou où il apprend les bases du socialisme et du marxisme. Quand il retourne en Chine il trouve un climat des guerres entre nationalistes et communistes. Deng commence son parcours politique et devient tôt connu quand il est nommé le responsable du sud-est de la Chine par Mao. Il est chargé d’éliminer les nationalistes. Il est dirigeant régional et l’unique lieder local qui a un contact direct avec Mao. En 1952, à Pékin, Deng et nommé par Mao vice premier ministre de l’économie et il commence sa marche vers le pouvoir. En Chine les politiques dictés par Mao sont misés en acte, une d’elles est la Champagne des Cent Fleurs , qui cause une catastrophe : pauvreté et morte. À cause de son erreur Mao se retire au sud, où il reste secrétaire général du PCC. Deng commence à formuler son opinion qui encourage la privatisation des terres, qui se pose contre l’idéologie communiste. Dans un discours public il soutient que pour sauver la Chine, le bon sens doit prendre le pas sur l’idéologie et il prononce la célèbre phrase « peu importe que le chat soit noir ou gris pourvu qu’il attrape les souris ». Il commence à prendre les distances des idéologies trop restrictives du communisme et après le lancement de la Révolution Culturelle  par Mao, Deng se déclare publiquement contraire. Il commence une propagande contre lui qui est vu comme figure capitaliste et il est condamné à l’exile à la campagne pour 5 ans, quand la juste occasion se présente, et il décide de retourner actif en politique et il se déclare une autre foi allié avec Mao. En tant en Chine l’industrie est paralysée, les universités sont fermées et rien ne marche pas dans la juste direction. En 1975 Deng et Zhou Enlai  proposent les quatre modernisations et la situation économique s’améliore. Il y a des opposer mais après la mort de Mao en 1976, Deng prend le pouvoir et il assume la fonction de maxime autorité du pays. Avec quelques difficultés Deng commence sa politique vers « le socialisme du marché »  qui transforme la Chine d’un pays arrêté à la puissance qu’elle est aujourd’hui. Deng soutient que les technologies, les capitaux étrangers et les méthodes de gestion économique occidentales ne sont pas capitalistes pas socialistes, ils sont neutres. Il ne regrette pas l’initiative et l’enrichissement privés comme dans l’idéologie communiste car il soutient qu’ensemble ils permettent un enrichissement collectif. Il abandonne la lutte de classe car il soutient que la chose la plus importante est l’avancement économique du pays.
1.2.2 La concrétisation des idées innovatives de Deng
Deng Xiaoping, nouveau leader d’Etat, propose une politique d’ouverture du marché chinois très graduelle vers l’étranger qui comporte un affaiblissement du rôle de l’Etat dans l’économie.   La politique qu’il propose est définie comme « socialisme du marché ».  Il soutient que grâce à l’ouverture du marché, la Chine peut bénéficier des échanges avec l’étranger et de l’affluence du capital. Deng propose avec Zhou une politique de modernisation dans les quatre secteurs les plus importantes : agriculture, technologie-science, défense nationale et industrie.  Cette modernisation permettent des pas en avant dans les domaines qui encouragent l’économie.
Afin de réaliser ce changement il met en acte une réforme qui se déroule en deux stades. La première partie de la réforme touche principalement le secteur agricole. Avant, l’état contrôlait et décidait les prix des produits et la quantité de la production pour permettre le rationnement et la distribution des biens. Ensuite, l’équilibre du marché est toujours plus orienté vers un dualisme des prix où les prix sont dictés par l’Etat et par l’interaction entre offre et demande du marché.  Pour arriver à ça, un pas fondamental est l’instauration des contrats familiaux : dans les campagnes l’agriculture est libéralisée mais la possession des terres est encore de l’Etat. La terre est partagée en petites parties et est attribuées aux familles paysannes. La famille qui travaille la terre s’engage à livrer des quotas de production à l’autorité locale à des prix fixés par l’Etat et à verser des impôts agricoles à l’Etat. Dans ce manière les terres ne sont pas totalement privatisées on a seulement privatisé leur usage.  De toute façon, chaque changement dans l'utilisation des terrains nécessite l'approbation du gouvernement local dont ils dépendent. Seulement en 1982 les paysans acquirent le droit de vendre le surplus de céréales sur le marché à des prix libres déterminés par l'interaction de l'offre et de la demande. En 1985, l'Etat supprime définitivement les livraisons obligatoires.  Le retour à l’exploitation familiale et la décollectivisation des terres a permis la livraison de beaucoup de main d’ouvre des campagnes aux villes.  A la fin des années 1960, après la révolution culturelle, 15-20 millions de jeunes sont déportés des villes aux zones rurales et obligés de travailler la terre.  Avec la décollectivisation des terres, qui a permis la circulation des personnes, on assiste à un fort exode rural et au cours des années 1980, environ 50 millions des personnes migrent des campagnes vers les zones plus urbanisées. 
La deuxième phase de la réforme encourage la privatisation des entreprises au moyen d’une réforme du secteur bancaire, des banques plus spécialisées sont créées ou rétablies.  Et par conséquence le pouvoir économique se déplace toujours plus vers les particuliers. La réforme touche surtout les entreprises et les industries. Depuis 1979, elles sont autorisées à conserver une partie des bénéfices au-delà de l'objectif fixé. En plus, c’est introduit une contractualisation des relations avec l'autorité de tutelle qui règle la répartition des bénéfices.  Elles n'ont pas d'autonomie en matière de fixation des prix, du niveau d'emploi, du niveau d'investissement même si leur pouvoir est accru.  En octobre 1984 est lancée la réforme des prix industriels : le dualisme des prix après la mise en application initiale à l'agriculture, se généralise à de nombreux biens de consommation, puis de production.  
Un autre changement fondamental qui suive la « politique de la porte ouverte » et qui a permis le développement chinois, consiste dans la création des zones de libre marché avec le but d’expérimenter l’ouverture du marché.  En 1979, sont créés quatre zones économiques spéciales, appelées ZES, qui sont situées dans la côte sud dans les deux provinces Guangdong et Fujian et elles sont Shenzhen, Zhuhai, Shantou et Xiamen.  Dans les ZES est proposée une réduction fiscale qui vise à attirer les investissements étrangers et les entrepreneurs. En plus, l’Etat propose mesures permettant aux entreprises de réduire leur coût d’investissement et de production. Ces mesures comprenaient principalement des réductions ou des exonérations totales de taxes, et des aides financières d’installation.  Vu le succès de ces zones spéciales, au cours des années ’80-’85, les ZES se sont multipliées et aujourd’hui elles sont quinze.   La création des ZES a permis une affluence de capital étranger vers les zones côtières et la création d’un terrain fertile pour la naissance des petites moyennes entreprises qui produisent biens manufacturés destinés à l’exportation. On vit un développement extraordinaire des villes et une urbanisation accélérée de toute la partie orientale du pays. Dans les années ’80-’85, en Chine il y a une forte augmentation des investissements provenant de l’étranger. Ces investissements arrivent en grand partie du Japon et des Dragons asiatiques  et ils sont destinés à l’industrie du secteur textile et des manufactures.  La Chine est devenue un lieu stratégique pour les investissements car le bas coût de main d’ouvre et le bas coût de production permettent un prix très compétitif pour les exportations.  La création des ZES et la progressive ouverture du marché chinois à l’étranger ont permis l’affluence du capital nécessaire pour garantir la naissance des plusieurs entreprises.

Dans l’image en haut on remarque la naissance progressive des ZES et leur localisation géographique. Les villes ou sont localisées les ZES sont celles qui se sont développées plus.
Comme on peut voir dans les images au-dessous elles correspondent aux zones où se concentre la plupart de la richesse et où est faite la plupart des IDE.

La Chine représente un lieu attractif pour les investissements grâce aux bas coûts de production. En 1978 la Chine reçoit le 0.8\%  des investissements directes l’étranger au niveau mondial et en 2005 elle rejoint le 6\%.  La naissance des nouvelles entreprises privées augmente la concurrence à l’intérieur du pays qui incite l’innovation et le progrès technologique.  Au même moment le développement du secteur financer permet l’offre du capital pour la constitution des entreprises et permet aussi aux entreprises le financement interne grâce au marché des titres.  L’ouverture du secteur financier à l’étranger fait augmenter la compétitivité et appuie le développement de l’économie chinoise. On assiste aussi à une forte perte d’importance dans l’économie du secteur primaire qui est remplacé par les secteurs des services et industriel. Selon le FMI le pays connaît depuis le début des réformes en 1979 un taux de croissance en 2003 et en 2004 de 7,5\% et après il reste en moyenne de 8,8\% par an entre 1990 et 2001. 
Entre le 1978 et le 1995 le PIB multiplie de plus que trois fois, la croissance démographique ralentit et le revenu par habitant se multiplie par 2,7 et le PIB par habitant augmente de 6\% par an.  

Le graphique montre la croissance du PIB. On remarque que pendant la période 1890-1952 elle était faible mais vers la fin du ‘900 on assiste à une super croissance surtout après les réformes du 1978.

Le graphique montre la structure du PIB chinois. On remarque une progressive perte d’importance du secteur agricole qui est remplacée surtout par une augmentation du secteur industriel. 

Le développement économique qui a lieu en Chine pendant les années ’80 a naturellement influencé la vie des personnes. On peut affirmer que les changements sociaux qui se sont passés n’ont pas été les mêmes pour toutes les personnes. La croissance économique a apporté plus de richesse dans le pays, mais elle n’est pas distribuée en manière équitable dans toutes les régions et pour toute la population.  Aussi le pouvoir des acteurs privés, qui a grandi en manière exponentielle, a créé des grosses disparités entre classes des travailleurs. Se crée une société différente de celle maoïste dans laquelle la position de l'individu était héréditaire et surtout fixée une fois pour toutes. 
Surtout dans un pays comme la Chine, dont la surface mesure 9 598 000 km2, l'existence d'inégalités entre les régions est évidente.   En plus il faut prendre en compte le fait que la Chine est un pays fédéral dans lequel les différentes provinces ressentent différemment la mondialisation.  Après l’ouverture à l’étranger du marché chinois on assiste à la naissance d’une énorme disparité entre les villes, qui reçoivent toujours plus capitaux, et les zones rurales, qui perdent toujours plus d’importance.  En Chine on assiste à une métamorphose géographique. La décollectivisation des terres permet à des millions de personnes de laisser le travail agricole pour chercher des conditions de vie meilleures dans les villes. Il a lieux un exode rural qui fournit la main-d’œuvre nécessaire pour le développement toujours croissant des villes surtout côtières. 

Le graphique montre comment la population de chine se distribue entre villes et campagne pendant les ans. On remarque une augmentation toujours plus forte de la population urbaine à partir du 1979 et une diminution parallèle de la population rurale. En 2009 la population est parfaitement répartie entre les deux zones et dans les ans suivants on constate que la population urbaine est majeure de celle rurale.
Parallèlement à ce phénomène on constate un développement majeur des villes, en effet, en 2007, en Chine, le 24\% des personnes travail dans le secteur secondaire, le 45\% dans le premier et le 31\% dans le tertiaire ; mais le pourcentage change si on considère les différentes zones du pays.  C’est-à-dire que dans les villes côtières le taux des travailleurs du secteur industriel est mangeur à la moyenne (Shanghai 39\% ou Zhejiang 42\%) et dans les pays plus internes du pays c’est le contraire (Tibet 9\% ou Yunnan 10\%). Le secteur tertiaire ne présente pas très grandes disparités entre régions à l’exception des deux grandes villes Pékin (68\%) et Shanghai (54\%).  La présence s’un nombre plus haut des travailleurs du secteur secondaire par rapport au primaire dans les zones côtières est un indicateur du développement. 
On peut affirmer que sont les zones côtières du sud-est, où sont situées la plupart des entreprises, qui trainent l’économie de la Chine et il s’agit de la zone qui a bénéficié plus du changement vécu par la Chine.  En effet, dans ces zones côtières le revenu moyen est naturellement majeur par rapport au nord du pays, où la majorité des entreprises appartient encore à l’Etat et le peu de richesse produite provient des terres.  En 2007 on constate que le revenu des résidentes urbaines chinois est en moyenne 3.55 fois plus élevé que ceux des résidents ruraux, cet écart était de 1.7 en 1984.  Aussi le revenu des paysans augmente, mais pas vite que ceux des travailleurs urbains .

Le graphique montre comment la disparité entre zones urbaines et rurales est accrue après l’ouverture. La naissance des nouvelles professions due au développement des secteurs industriel et des services apporte une augmentation du revenu dans les villes mais n’est pas le même pour les travailleurs des campagnes. 

\subsection{Les conséquences sociales dues à un développement très rapide}
Le développement économique qui a lieu en Chine pendant les années ’80 a naturellement influencé la vie des personnes. On peut affirmer que les changements sociaux qui se sont passés n’ont pas été les mêmes pour toutes les personnes. La croissance économique a apporté plus de richesse dans le pays, mais elle n’est pas distribuée en manière équitable dans toutes les régions et pour toute la population.  Aussi le pouvoir des acteurs privés, qui a grandi en manière exponentielle, a créé des grosses disparités entre classes des travailleurs. Se crée une société différente de celle maoïste dans laquelle la position de l'individu était héréditaire et surtout fixée une fois pour toutes. 
Surtout dans un pays comme la Chine, dont la surface mesure 9 598 000 km2, l'existence d'inégalités entre les régions est évidente.   En plus il faut prendre en compte le fait que la Chine est un pays fédéral dans lequel les différentes provinces ressentent différemment la mondialisation.  Après l’ouverture à l’étranger du marché chinois on assiste à la naissance d’une énorme disparité entre les villes, qui reçoivent toujours plus capitaux, et les zones rurales, qui perdent toujours plus d’importance.  En Chine on assiste à une métamorphose géographique. La décollectivisation des terres permet à des millions de personnes de laisser le travail agricole pour chercher des conditions de vie meilleures dans les villes. Il a lieux un exode rural qui fournit la main-d’œuvre nécessaire pour le développement toujours croissant des villes surtout côtières. 

Le graphique montre comment la population de chine se distribue entre villes et campagne pendant les ans. On remarque une augmentation toujours plus forte de la population urbaine à partir du 1979 et une diminution parallèle de la population rurale. En 2009 la population est parfaitement répartie entre les deux zones et dans les ans suivants on constate que la population urbaine est majeure de celle rurale.
Parallèlement à ce phénomène on constate un développement majeur des villes, en effet, en 2007, en Chine, le 24\% des personnes travail dans le secteur secondaire, le 45\% dans le premier et le 31\% dans le tertiaire ; mais le pourcentage change si on considère les différentes zones du pays.  C’est-à-dire que dans les villes côtières le taux des travailleurs du secteur industriel est mangeur à la moyenne (Shanghai 39\% ou Zhejiang 42\%) et dans les pays plus internes du pays c’est le contraire (Tibet 9\% ou Yunnan 10\%). Le secteur tertiaire ne présente pas très grandes disparités entre régions à l’exception des deux grandes villes Pékin (68\%) et Shanghai (54\%).  La présence s’un nombre plus haut des travailleurs du secteur secondaire par rapport au primaire dans les zones côtières est un indicateur du développement. 
On peut affirmer que sont les zones côtières du sud-est, où sont situées la plupart des entreprises, qui trainent l’économie de la Chine et il s’agit de la zone qui a bénéficié plus du changement vécu par la Chine.  En effet, dans ces zones côtières le revenu moyen est naturellement majeur par rapport au nord du pays, où la majorité des entreprises appartient encore à l’Etat et le peu de richesse produite provient des terres.  En 2007 on constate que le revenu des résidentes urbaines chinois est en moyenne 3.55 fois plus élevé que ceux des résidents ruraux, cet écart était de 1.7 en 1984.  Aussi le revenu des paysans augmente, mais pas vite que ceux des travailleurs urbains .

Le graphique montre comment la disparité entre zones urbaines et rurales est accrue après l’ouverture. La naissance des nouvelles professions due au développement des secteurs industriel et des services apporte une augmentation du revenu dans les villes mais n’est pas le même pour les travailleurs des campagnes. 
\subsection{Le développement de la technologie en Chine et la naissance de Huawei}
Huawei est une entreprise qui nait en 1987 fondée par Ren Zhengfei  avec l’aide d’une équipe de dix personnes dans la ville de Shenzhen.  Il s’agit d’une société privée totalement possédé par ses travailleurs.  La société a le but de gérer les ventes d’un producteur de centraux téléphoniques privés dans la ville de Hong Kong et elle fait des activités de revente du matériel.    Mais grâce au recueil des informations sur le secteur, l’entreprise met en place un petit grouppe d’ingénieurs et elle commence à produire ses appareils de commutation.  En 1990 elle commence à produire équipement télécom pour hôtels et petites industries.  Dans les années ’90 grâce au succès, l’entreprise produit sa centrale digitale qui est une des plus importantes et puissantes sur le marché de la Chine. Au début l’entreprise construisait des réseaux de téléphonie fixe, et en 1997 elle commence à produire appareils pour la téléphonie mobiles et solutions wireless GSM . Pendant ces années ces produits sont en plein développement. Pendant les années 2000 elle investit autour le 10\% de son chiffre d’affaire dans le secteur de recherche et développement et elle travail en partie en collaboration avec les universités très performantes dans les domaines des technologies . En effet le 45\% du personnel qui travaille pour l’entreprise est occupé dans la R&D . Le premier centre de recherche à l’étranger est ouvert en Inde en 1999, l’année successif il est ouvert un autre centre de recherche en Suède et en 2001 ils en sont ouverts quatre dans les Etas Unis . En 2004 l’entreprise signe un important contrat de 25 millions de dollars avec une société européenne .  C’est comme ça que Huawei affirme son expansion internationale. L’entreprise commence à produire appareil technologiques et appareils téléphoniques (smartphones) et grâce aux investissements dans le secteur des recherches elle continue à proposer nouveaux produits. En 2008 elle construit un réseau mobile sur large échelle en Amérique du nord et elle rejoindre le troisième range pour le chiffre du marché mondial dans les appareils de réseau mobile . Huawei est une des primes entreprises qui ouvre les infrastructures de 4G et elle continue à proposer nouveaux produits technologiques et en brevette toujours plus . Tout cela permet à l’entreprise d’opérer au niveau international et d’arriver toujours plus en haut dans l’échelle du marché mondial. En 2017 elle est une des primes entreprises qui conduit des tests pour la commercialisation de service de 5G . La vente de smartphones devienne toujours plus importante et en 2017 Huawei est un des trois producteurs de smartphones un niveau mondial avec Samsung et Apple . Dans les derniers ans elle conduit recherches pour l’application de l’intelligence artificiel dans les dispositifs technologues et de téléphonie.  Tout cela lui permet de devenir l’entreprise que on connait aujourd’hui comme Huawei, un fournisseur leader au niveau mondial des infrastructures pour les technologies de l’information et de la communication (TIC). 
On peut affirmer que le développement de Huawei est en grand partie dû à la recherche des nouvelles technologies qui lui permet de proposer produits innovatifs sur le marché. Tout cela est strictement lié au développement chinois car la croissance économique apporte les capitaux nécessaires aux recherches et, comme on va voir dans le chapitre suivant, la politique que Deng propose touche aussi aspects liés à l’instruction et aux recherches scientifiques. 
En effet, comme on a déjà dit Huawei nait en 1987 à Shenzhen, il s’agit d’une ville côtière qui possède une position stratégique, elle collègue Hong Kong avec le reste de la Chine. La ville se trouve dans la province de Guangdong au sud de la Chine, qui est une des zones économiques spéciales crées en 1979.  Grâce à cela la ville de Shenzhen passe de 20'000 à plus de 10 millions d’habitants en vingt ans et le revenu moyen par habitant est passé de 323 yuans en 1980 (la moyenne nationale était de 321) à 3910 en 1993 (la moyenne nationale était de 2100) et elle se place en première position parmi toutes les villes de la Chine, en 2011 quand le revenu moyen rejoint le chiffre de 36505 yuans.   Cette ville est devenue siège d’entreprises connues du secteur technologique comme par exemple Tecent Holding . En effet la ville a subi une croissance des nouveaux marchés et des innovations des produits et tout cela est permis par le développement des technologies qui se produit en Chine après les réformes de Deng.
Dans les années 1950 la Chine établie des liens de collaboration avec la communauté scientifique en Union Soviétique . La collaboration avec l’Union Soviétique permet une introduction massive des sciences des technologies en Chine qui contribuait à la création d’une baise industrielle solide.  Le bouleversement politique qui suive le Révolution Culturelle jusqu’ à la fin des années ’70 isole la communauté scientifique chinoise.  Mais avec la politique de la porte ouvert de Deng la Chine réalise progrès importants dans le domaine scientifique ; la technologie favorise les IDE et au même temps les IDE apportent le capital qui permet le développement technologique.   En plus l’augmentation des connaissances techniques permet une productivité supérieure et facilite la différentiation des produits dans le secteur industriel ; tout cela apporte à profits majeurs qui permettent aux entreprises l’achat des nouvelles technologies d’étranger . 
La progressive privatisation des entreprises (v. chapitre 2.3) et l’affaiblissement du rôle de l’Etat (v. chapitre 2.3)  dans l’économie permettent une augmentation de la compétitivité qui puisse les investissement de la part des entreprises dans le R&D toujours majeur.  En plus dès années ’80 il naissent différentes programmes des recherche subventionnés par l’Etat comme «le programme TORCH » qui nait en 1988 et qui s’intéresse du développement de haute technologie et sa application industrielle.  De 1994 à 1998 les dépenses consacrées à projets de R&D dans les universités passent de 2,4 à 5,16 milliards de yuans et en même temps les dépenses consacrées dans le domaine des sciences et technologies passent de 3,7 à 7,37.

