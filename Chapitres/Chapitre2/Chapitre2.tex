\section{Le rôle de l’Etat chinois dans la globalisation}
Dans ce chapitre on va approfondir la thématique du rôle du parti communiste chinois, et donc de l’Etat chinois, dans le développement économique, social et politique que la Chine a eu dans les dernières décennies. 
\subsection{Entre 1911 et 1949 : la prise de pouvoir du parti communiste chinois}
Le parti communiste chinois (PCC) est créé le 1er juillet 1921, journée où se déroule le premier Congrès du Parti, mais il est apparaît de facto l’année avant, le 1er août 1920. Il est formé dans un contexte où la Chine est en train de faire face à des grands changements au niveau politique mais aussi social et historique. En effet une dizaine d’années avant la date de création du PCC, plus précisément en 1911, a lieu ce que les historiens appellent révolution Xinhai. Cette révolution, qui avait comme but de rendre un paye comme la Chine, qui était considéré attardé par rapport au reste du monde, plus moderne et fort, était guidée par le politicien révolutionnaire Sun Yat-sen. Mais ce but n’est pas accompli et la situation de la nation devient encore pire. Il est nécessaire d’attendre jusqu’aux réformes de Deng Xiaoping en 1978 pour que cela soit possible. Ce bouleversement conduit en 1912 à la chute de l’empire chinois et à la proclamation de la république fondée par Sun Yat-sen. La Chine, qui avait été depuis 221 avant J.C une monarchie, devient donc une république. Le parti qui gouverne la république chinoise dans cette période a été constituée par le même Sun Yat-sen,  et il s’agit d’un parti nationaliste appelé Guomindang (GMD) qui existe encore aujourd’hui à Taiwan. 
Dès 1916, la république se divise en différentes régions gouvernées par des « seigneurs de la guerre » qui se disputent pour gagner les plus grandes territoires chinois en créant des conflits dans toute la nation.  Cela arrive donc dans des années très difficiles pour tout le pays quand on voit la naissance du PCC, un parti fondé aussi grâce à la révolution d’octobre de 1917 en Russie.  En effet la révolution russe, qui conduit à la constitution de la république soviétique, est prise comme point de référence par Chen Duxiu pour la fondation du parti.  Les idéologies marxistes et léninistes poussent les communistes chinois à faire des changements dans leur société, mais ils vont les adapter aux conditions chinoises. Les idéologies qui inspirent les communistes chinois concernent surtout le fait de renier les idées et les principes confucianistes sur lesquels l’organisation au niveau social du pays s’est fondée pendant des siècles. En effet la Chine se base sur les enseignements du philosophe chinois Confucio dès sa naissance en 551 avant J.C, doctrines qui se fondent sur des normes sociales qui peuvent permettre de discipliner les relations humaines et donc de garantir l’ordre social avec des hiérarchies soit familiales soit sociales. Les communistes pensent donc que pour résoudre les problèmes des chinois il n’est pas suffisant de changer le régime politique, mais il faut aussi révolutionner la mentalité et laisser tomber les vieux principes d’organisation de la société pour devenir un pays fort et uni. Chen Duxiu, en parlant de ce qui est nécessaire de faire pour sauvegarder la Chine, dit que : « J’aimerais mieux assister à la ruine de notre « quintessence nationale » qu’à l’extinction définitive de notre race, en raison de son inaptitude à survivre ».  
Dès 1921, le PCC augmente toujours plus son pouvoir sur le territoire et va se disputer avec le GMD jusqu'à 1948, même s’il y a des brèves périodes d’alliance entre les deux partis pour combattre des ennemis communs comme par exemple pour mettre fin à la période des « seigneurs de la guerre ».  Les conflits entre PCC et GMD sont plusieurs et ils sont aussi très sanglants, avec de nombreuses victimes. Un exemple de ces conflits peut être le « coup de Shanghai » en 1927, une journée qu’on peut aussi appeler comme « massacre des communistes » et qui lance la persécution contre les communistes des années suivantes.  Dès 1945, on voit comment le PCC devient un prétendant au pouvoir sur toute la nation ; en effet, en 1946 entre les deux partis recommencent les conflits qui éclatent in une guerre civile vaincue par le PCC en 1949.  Le 1er octobre 1949 Mao Zedong, le président du parti depuis 1935, proclame la république populaire de Chine,  en faisant devenir cette date la fête nationale de la nation. Avec la déclaration de la république populaire de Chine, le PCC devient le parti unique qui détient le pouvoir et donc le seul parti au gouvernement

\subsubsection{Entre 1949 et 1976 : les années de Mao Zedong }
Le gouvernement créé en 1949 doit tout de suite faire face à des problèmes géants qui tourmentent la population et le pays entier. Les conflits de la période précédente ont créé de la dénutrition sur tout le territoire, la détruite des infrastructures suite à la guerre et des graves problèmes de santé publique et d’analphabétisme. L’économie du pays, qui dans cette période se base principalement sur la production agricole, est en ruine avec un PIB très bas et une monnaie qui perde de valeur de plus en plus.  Ce thème sera approfondi dans le prochain chapitre. Donc la Chine en 1949 est un pays détruit, mais le nouveau gouvernement avec Mao Zedong et tous ses alliés vont essayer de résoudre la situation avec des réformes économiques et agricoles et des améliorations au niveau de santé et d’éducation.  Les reformes sont initialement agricoles et concernent surtout la possession des terres agricoles. En effet plus ou moins le 45 \% des terres arables ont été redistribuées parmi les grands propriétaires fonciers et ceux qui possédaient que quelques acres ou même rien.  L’Etat puisse les agricultures à la création des équipes d’entraide à l’intérieur des régions chinoises. Cela fait en sorte que en 1952 le 39\% des propriétés agricoles appartiennent aux équipes d’entraide en conduisant les prix à être stables et donc aussi aux niveaux de production agricole à être comme avant la guerre civile.  Après avoir résolu la crise agricole, le PCC met en place des plans quinquennaux entre 1953 et 1957 qui concernent le développement à la fois sociale et économique. Entre 1953 et 1957 on parle su premier plan quinquennal de la république populaire de Chine, mais plus ou moins chaque décennie il y a un nouveau plan. En 2016 est commencé le treizième plan, qui va terminer en 2020.  Chaque fois les plans se concentrent sur des questions actuelles et très différentes entre eux. Pour ce qui concerne le premier plan du 1953 le programme se concentre surtout sur le développement industriel. Pour faire en sorte que cela marche le PCC adopte le modèle économique soviétique, qui se base sur la possession de la partie de l’Etat de la majorité de secteurs économiques et une économie planifiée et centralisée. Pour faire cela la Chine est soutenue par l’union soviétique à la fois financièrement mais aussi par la main-d’œuvre. La réforme fonctionne de façon étonnante, la situation industrielle est améliorée.  Un autre fait qui cause des problèmes est que l’organisation du gouvernement n’est pas encore stable, en effet il est nécessaire d’attendre jusqu'à 1954 pour avoir une première constitution de la république populaire de Chine et jusqu’à ce moment-là le pays est guidé à travers des congrès et des réunions. Il est important de comprendre que même si le PCC est le seul parti au pouvoir le parti nationaliste existe encore à l’intérieur de l’État, mais il n’a pas de pouvoir. En fait les nationalistes font aussi partie du gouvernement, ce qui est dû au fait que la population qui soutient le PCC n’est pas assez instruite pour les emplois que le parti offre.  En 1958 avec des réformes et beaucoup de changements la situation s’améliore énormément par rapport à 1949 : l’économie est stable avec la production agricole qui est retournée à des niveaux normaux, la population est toujours plus instruite et aussi le niveau général de santé s’améliore avec une croissance de l’espérance de vie à 57 ans. C’est dans ce contexte de bien-être général que Mao Zedong prend la décision de faire encore plus, avec un programme économique qu’il appelle « Grand Bond en avant », nom qui décrit le grand saut en avant au niveau économique dont il rêve.  Ce programme a comme but de faire avancer tous les secteurs de production de la Chine jusqu’à arriver au niveau de la Grand Bretagne dans les 15 années suivantes. Pourtant, le programme ne fonctionne pas, probablement étant trop ambitieux et conduisant aux changements trop forts.  Ces réformes causent dans le pays une crise alimentaire et la mort de plus ou moins 30 millions de personnes en quatre ans.  En 1959, vu la crise que son programme avait créé, Mao Zedong laisse son poste du président du gouvernement, en abandonnant ses alliés dans une situation de grande crise politique mais surtout économique.  En 1966 Mao Zedong retourne au pouvoir du PCC et commence sa dernière campagne politique, la « Révolution culturelle » qui avait pour but de supprimer le vieil ordre social pour le remplacer avec un nouveau qui était surtout socialiste.  Encore une fois il suscite des conflits, surtout entre la vieille génération et les plus jeunes membres du parti. Ces derniers vont suivre les indications de Mao de détruire tout ce qui peut être considéré comme vieux : les habitudes, les coutumes, la culture et les idées. Ces groupes vont créer les gardes rouges, qui vont soutenir et transmettre au peuple la pensée de Mao de renouvèlement de la société. Ces gardes vont persécuter et même tuer tous les opposants au PCC ou au socialisme, en causant plus ou moins un demi-million de morts et une série de problèmes psychologiques et physiques à cause des violences infligées aux millions de gens.  En 1969, Mao déclare que la révolution a fonctionné et en effet la culture semble révolutionnée. Mais même si la révolution a eu un résultat positif, il y a eu des pertes économiques et agricoles importantes qui ont causé des problèmes dans la période à suivre. En effet il est nécessaire d’attendre jusqu’aux années quatre-vingt pour que l’économie chinoise commence une période de reprise après les réformes de Mao.  Ce dernier va mourir en 1976, après une dizaine d’années pendant lesquelles ses choix au niveau politique vont causer encore d’autres problèmes pour la nation.  
Les années de Mao ont été fondamentales pour le développement du pays, est grâce à lui que la Chine a pu se moderniser et se développer surtout dans les secteurs industriels et agricoles, même si le progrès a été instable et été concentré surtout sur la révolution du pays.  Mais les opinions concernant son rôle à l’intérieur du développement de la Chine sont discordantes, vu que ses politiques ont causé des grands problèmes au niveau économique avec des pertes financières importantes et des millions de morts.  La majorité des chinois voit Mao comme un symbole, mais à l’extérieur du pays on trouve des descriptions pas si positives qui le décrit comme un dictateur l’égal de Staline et Hitler. Le problème est que la documentation concernant ses actions est encore aujourd’hui cachée, donc n’est pas encore possible faire une analyse objective du son opérée. Ce qu’est certain est que la période maoïste a conduit la Chine vers la grande croissance économique et démographique que tout le monde connaît, mais que ce développement a coûté beaucoup en termes de vies humaines. 

\subsubsection{Entre 1976 et 2013 : les réformes de Deng Xiaoping et la croissance économique}
La personne qui prend sa place, Deng Xiaoping, qui commence une période caractérisée par des réformes et l’ouverture des marchés, une période qui a été déjà analysée de façon plus approfondie dans les chapitres précédents. C’est après ses réformes de 1978-1979 que l’idéologie chinoise prend le nom officiel de « socialisme avec caractéristiques chinoises », un terme qui explique la nécessité d’adapter les idées du communisme européen à la situation chinoise.  Vers la fin de la période de pouvoir de Deng, qui meurt en 1997,  la Chine entre dans des années de forte croissance économique qui conduite son PIB à être à la troisième place au niveau mondial en 2008 avec 4'500 milliards de dollars, donc avec une augmentation de 261\% depuis le 2000.  La croissance économique continue aussi après la mort de Deng, qui est remplacé par un autre membre du PCC Jiang Zemin, qui continue les travaux de Deng en ce qui concerne l’économie. Les réformes mises en place par lui sont donc faites en suivant le modèle de Deng et servent surtout pour faire en sort que la croissance économique ne s’arrête pas. 
Après la mort de Mao le PCC subit des changements importants, surtout pour ce qui concerne la formation des membres. En fait, avant les réformes de Deng à l’intérieur du parti, la moitié des membres du PCC avait seulement un diplôme de l’école primaire ou était même analphabète.  Grâce à Deng en 1986 plus d’un million de vieux membres du PCC partent à la retraite, ce qui fait en sorte que leurs places sont prises par des jeunes avec un diplôme universitaire. Cela conduit le parti communiste à être composé par des fonctionnaires toujours plus formés et conduit le nombre de membres avec une formation scolaire à augmenter d’une année à l’autre. Un autre changement a lieu dans les années 2000 quand le président Jiang décide que les trois groupes sociaux qui ont toujours été exclus du parti peuvent y prendre part : les entrepreneurs, les scientifiques et les intellectuels.  Ce changement fait en sorte que le nombre d’inscription au PCC augmente de 10\% de 2002 à 2007.  C’est important de noter que l’inscription au parti communiste était et est toujours possible seulement à un nombre resserré de gens et que des contrôles se font sur les candidats. Le numéro d’inscrits au PCC augmente de plus en plus : en 1956 les membres du parti étaient environ 10 millions mais en 2009 le numéro est arrivé à 75 millions de personnes appartenant au parti. Ces chiffres conduisent le PCC à être, en 2009, le plus grand parti unique au monde.  En 2003 Jiang Zemin laisse sa place de président et est remplacé par Hu Jintao.  Le changement de président n’a pas créé des problèmes mais le nouveau président prend le pouvoir à nouveau dans une période de tension et de bouleversement surtout au niveau international : une nouvelle guerre en Iraq, une crise née à cause d’une nouvelle politique nucléaire faite par la Corée du Nord et une épidémie de SARS dans les zones rurales chinoises. Ceux-ci mais pas seulement, ont été des facteurs problématiques à gérer pour le président. Mais avec le soutien du premier ministre Wen Jiaba pour ce qui concerne la création des réformes nécessaires à résoudre la situation de crise, la situation se stabilise et la continuité dans la croissance économique du pays est assurée.  Xi Jinping succède au président en exercice en 2013.

\subsection{Le rôle de la monnaie }
La monnaie a un rôle fondamental au sein de chaque pays à la fois économiquement et socialement, et la Chine n’en est pas moins à cet égard. La devise nationale chinoise a grandement influencé la société du pays, et dans ce chapitre les principaux passages qui ont conduit le yuan à être ce que nous connaissons aujourd’hui sont expliqués.
\subsubsection{L’histoire de la devise nationale chinoise : renminbi et FEC}
La devise nationale de la république populaire de Chine est depuis la fin du 1951 le renminbi, nom qui signifie « monnaie du peuple ».  Le yuan, nom utilisé souvent pour faire référence à la monnaie chinoise, est en réalité l’unité de compte du renminbi.  Donc le yuan est le nom de référence au niveau international et est aussi la référence pour ce qui concerne le taux de change. En janvier 2019 le taux de change entre yuan et franc suisse est de 0.14446 yuan pour 1 franc suisse. La première série de renminbi est apparue en Chine en 1948 à la suite de la fondation de la Banque Populaire de Chine,  pendant la guerre civile entre le PCC et le Koumindang déjà expliquée dans le chapitre précédent. Vu que la Banque Populaire de Chine était sous le domaine du PCC, les renminbis étaient initialement la monnaie du parti communiste.  En effet, le Koumindang avait une devise personnalisée qui a disparu l’année suivante quand le PCC à pris le pouvoir. Après la victoire du PCC la Banque Populaire de Chine devient la banque centrale du pays, et le renminbi est reconnu comme la devise officielle de l’Etat en 1951. Le système monétaire a également été unifié afin de réduire l’inflation créée pendant la guerre civile.  

En 1955 une nouvelle série de renminbi est émise, avec un taux d’échange de 10'000 vieux yuans pour 1 nouveau yuan.  En 1962 il y a une troisième nouvelle série de yuans qui est émise par la banque centrale de Chine, et pour la première fois les billets sont en couleurs. Ensuite, après les réformes de Deng deux autres séries de yuans, la première en 1987 et la dernière en 1999 ont été émises.  Les changements dans ceux dernières sériées sont surtout structurelles, comme par exemple la représentation de Mao Zedong sur tous les billets ou l’utilisation d’un encre spécial.  Jusqu’au 1er janvier 1995  en Chine il y a deux monnaies en vigueur :  le yuan, qui pouvait être utilisé seulement par les chinois, et le Foreign Exchange Certificate (FEC), qui était pour les touristes, les diplomatiques et les travailleurs étrangers.  Les FECs entrent en vigueur le 1er avril 1980 et leur but était de réduire les mouvements des étrangers sur sol chinois vu l’augmentation des visiteurs étrangers après l’ouverture des marchés de 1978. Les FECs étaient utilisables seulement dans certains hôtels et dans certains magasins qui étaient appelés « Friendship store ».  Ces derniers, auxquels l’entrée était interdite aux chinois, vendaient des produits de luxe aux étrangers comme les cigarettes de la marque Marlboro, les télévisions en couleurs ou les horloges suisses.  Les FECs étaient donc une monnaie privilégiée, qui créait tension à l’intérieur du pays vu qu’aussi les chinois voulaient en posséder pour pouvoir acheter les biens de luxe importés et pour pouvoir convertir la monnaie en dollars. Donc les chinois voulaient avoir des FECs pour faire tout ce qui n’était pas possible de faire avec le yuan dans cette période. Mais aussi les étrangers voulaient changer leur monnaie en utilisant le yuan pour pouvoir acheter plusieurs biens et services à l’intérieur de l’Etat chinois. Cela conduit à la naissance du marché noir pour la conversion de yuan en FECs, ou pour échanger de yuan en dollars. Avec ces échanges sur le marché noir les chinois pouvaient acheter des FECs et les étrangers achetaient des yuans en devenant libres de faire des achats dans tous les magasins chinois. Le prix d’achat à l’intérieur du marché noir était plus ou moins de 130 yuans pour 100 FECs, et de 9 yuans pour 1 dollar.  Au début de l’année 1995 les FECs sont abolis et cela conduit à des transformations à l’intérieur de l’Etat chinois. En effet, les « Friendship store » se transforment en simples magasins, dont certains vont maintenir leur dénomination initiale.  Le marché noir ne s’occupe plus des FECs mais seulement d’échanger les yuans en dollars. Cela parce qu’il faut attendre jusqu’à 1998 pour que les chinois peuvent échanger de façon légal leur argent en dollars.  En 1998 l’Etat chinois crée des nouvelles lois qui permettent aux chinois d’échanger les yuans en dollars, avec des règles en concernant la quantité d’argent que chaque individu peut acheter pour chaque voyage à l’étranger. Initialement le chiffre était de 2'000 dollars pour chaque déplacement à l’extérieur du pays, mais en 2007 le chiffre augmente jusqu’à 50'000 dollars chaque voyage.  En 2018 la Banque Populaire de Chine rappelle les renminbis de la quatrième série, et à partir de ce moment-là en Chine peuvent être utilisés seulement les renminbis de la cinquième série, celle de 1999. 

\subsection{L’influence de l’Etat sur les entreprises nationales}
Les politiques maoïstes des années cinquante ont eu une forte influence sur la privatisation des entreprises chinoises, et donc sur la possibilité du développement de l’économie privée du pays. Mais aussi les réformes de Deng Xiaoping ont porté à des changements dans le secteur privé chinois, en permettant aux entreprises de croître économiquement. Dans ce chapitre, les principales étapes du développement du secteur privé, qui ont permis la création des multinationales modernes, sont analysées.
\subsubsection{Le secteur privé chinois : histoire et développement}
L’influence que l’Etat, et donc le PCC, a sur les entreprises chinoises a changé au fil du temps. Dans les années cinquante, lorsque le PCC prit le pouvoir, presque toutes les entreprises présentes sur le sol chinois étaient publiques. Cela est dû surtout au fait que au moment de la prise de pouvoir du PCC la Chine est détruite, la guerre civile avec le Koumindang cause une série de problèmes géants dans tous les secteurs, comme a déjà été expliqué dans les chapitres précédents. Pour résoudre la situation le PCC, sous la direction de Mao Zedong, décide de rendre la plupart des entreprises publiques, dont la majorité étaient déjà publiques quand le pays était sous le contrôle du Koumindang pendant les années précédentes.  En effet, en 1952 seulement le 17\% des entreprises chinoises étaient privatisés, les autres étaient sous le domaine total du PCC.   Cela fait en sorte que les secteurs fondamentales pour l’économie étaient, et sont toujours, sous le domaine de l’Etat.  Ces secteurs ont changés au fil du temps, en passant par l’agriculture des premières années pour arriver au nouveau millénaire avec le secteur de l’aérospatiale et ce de l’automobilistique.  Jusqu’à 1978, année des réformes de Deng Xiaoping, le secteur privé n’existe pas sur le territoire chinois.  Des entreprises qui ne sont pas possèdes par l’Etat existent, mais il ne sont pas reconnues comme tels, et il diminuassent beaucoup pendant « le Grand Bond en avant » mais surtout pendant la révolution culturelle.  Donc est nécessaire d’attendre jusqu’aux modifications apporté par Deng Xiaoping pour voir des changements en ce qui concerne le secteur privé.  Après les réformes de Deng Xiaoping, plus précisément en décembre 1978, les entreprises privées sont reconnues comme partie intégrante du système économique chinois.  Initialement, ce type d’entreprise n’était accepté que dans les endroits où il y avait une mauvaise présence de sociétés d’Etat, donc dans des zones rurales avec une pénurie de services.  Dans les années suivantes les zones et les secteurs où les entreprises privées peuvent agir accroissent, et leur présence sur le territoire augmente de plus en plus. En 1988 est légalement acceptée la présence des grandes entreprises privées sur le territoire chinois, en faisant augmenter encore plus leur nombre.  Cela porte à la création des entreprises toujours plus grandes et puissantes, comme par exemple Huawei dont on parlera dans le prochain chapitre. Mais les multinationales chinoises plus grandes, qui travaillent surtout dans le secteur pétrolier et bancaire, ne sont pas privées.  En effet, elles sont contrôles par la Commission de contrôle et de gestion des biens publics (State-owned Assets Supervision and Administration Commission, SASAC),  qui est une institution sous la direction de l’Etat.  Cela fait en sorte que les multinationales dont l’Etat est le majeur actionnaire ont plus de facilité à croître grâce aux financements étatiques. 
\subsection{Le rapport entre Huawei et l’Etat}
Au fil des ans, de nombreux scandales ont perturbé le chemin de Huawei à la réussite. Ce chapitre explique la relation entre le PCC et la multinationale chinoise.
\subsection{Huawei, entreprise privée ou publique ?}
Huawei Technologies Co Ltd est une entreprise qui a été fondée en 1987 par Ren Zhengfei et est, depuis sa fondation, une entreprise privée.  Avant de créer l’entreprise Ren Zhengfei faisait partie de l’armée populaire de libération, qui est né en 1927 sous le nom d’armée rouge chinoise.  En 1993 Huawei commence à travailler avec l’armée populaire de libération de Chine, en construisant leur premier réseau national de télécommunication. Cette relation entre le parti communiste et l’entreprise a toujours créé des problèmes de fiabilité, surtout après le 1997 quand elle a commencé l’expansion envers les territoires étrangers. Huawei est accusée d’avoir eu le soutien économique de l’Etat chinois pour gagner le marché international, ce qu’elle nie.  Mais celle-ci n’est pas la seule accuse que Huawei a reçu au fil du temps, en effet dès 2003 la société a été accusée plusieurs fois de vol de propriété intellectuelle de certains entreprises, surtout dans les Etats-Unis.  Ce lien entre Huawei et le PCC est en train de causer beaucoup de problèmes à l’entreprise, surtout maintenant qu’elle est en train de développer un réseau wireless 5G, qui devrait être mis sur le marché en 2020.   Parmi les 170 pays où l’entreprise est présente, il y en a 5 (Etats-Unis, Canada, Angleterre, Australie et Nouvelle-Zélande)  où sa utilisation commence à être interdite et où le réseau 5G créé par elle ne pourra pas être utilisé.  La préoccupation de ces Etats, mais aussi de beaucoup d’autres comme l’Allemagne, consiste dans le fait que Huawei pourrait utiliser ses propres équipements technologiques pour espionner le gouvernement et la population de ces pays par la volonté l’Etat chinois.  La relation entre Huawei et le PCC et les accusations d’espionnage n’ont pas encore été clarifiées. Pour l’instant, ce ne sont que des accusations fondées sur rien de confirmé, aussi à cause de la censure présente sur le sol chinois qui ne permet la vision de certains documents qui pourraient clarifier la situation. Jusqu’au début de 2019, les accusations ont été désavoues par la société,  et les pays mentionnés ci-dessus continuent de rejeter l’utilisation des appareils technologiques de Huawei.  Il sera nécessaire d’attendre du temps pour comprendre comment la situation pourrait continuer. Pour le moment ce qu’on connaisse est que Huawei est devenue, au cours du deuxième trimestre 2018, la deuxième entreprise au niveau mondial par rapport à la vente de smartphones dans le monde en surclassant Apple.  Et pour confirmer la croissance de Huawei, même dans une situation de crise politique qui concerne ses actions, en 2018 une enquête a révélé que Huawei est la meilleure entreprise privée chinoise parmi les 500 examinées dans le sondage.  